%%%%%%%%%% SETTINGS %%%%%%%%%%%%%%%%%%%%%%%%%%%%%%%%%%%%%%%%%%%%%%%%%%%%%%%%%%%%

\documentclass[
	paper=a4,%
	pagesize,%
	8pt, fleqn,%
	headings=small,%
	notitlepage,%
	parskip=never]%
	{scrreprt}
	
% TODO: Edit metadata!
	
\newcommand{\mytitle}{Data Modeling}
\newcommand{\myauthor}{Shahwar Saleem}

%%%%%%%% Packages %%%%%%%%%%%%%%%%%%%%%%%%%%%%%%%%%%%%%%%%%%%%%%%%%%%%%%%%%%%%%%

\usepackage[top=0.5cm, left=0.5cm, right=0.5cm, bottom=0.8cm, landscape]{geometry}
\usepackage[T1]{fontenc}
\usepackage[utf8]{inputenc} 
%\usepackage[ngerman]{babel}
\usepackage[english]{babel} 
%\usepackage[charter, expert, cal=cmcal]{mathdesign}
\usepackage{array}
%\usepackage{listings}
\usepackage{enumerate}
\usepackage{ifthen,calc}
\usepackage{empheq}
\usepackage{wrapfig}
\usepackage{amsmath, amssymb, amsthm}
\usepackage{multicol}
\usepackage[svgnames,table,hyperref, dvipsnames]{xcolor}
\usepackage{graphicx}
\usepackage{multirow}
\usepackage{hyperref}
\usepackage{verbatim}
\usepackage{enumitem}
\usepackage{textcomp}
\usepackage{parskip}
\usepackage{blindtext}
\usepackage{floatflt}
\usepackage{float}
\usepackage{booktabs}
\usepackage[fulladjust]{marginnote}
\usepackage{longtable}
\usepackage{xtab,afterpage}
\usepackage{xcolor}
\usepackage{tikz}
\usepackage{accents}
\usepackage{tabularx}
\usepackage{hyperref}
\usepackage{bookmark}
\usepackage{siunitx}
\usepackage[most]{tcolorbox}
\usepackage{titling}
\usepackage[a]{esvect} % vectors, options a--h
\usepackage{tikz}
\usepackage{pgfplots}
\usepackage{physics}
\usepackage{booktabs}
\usepackage{empheq}
\usepackage{csquotes}
\usepackage{microtype}
\usepackage[style=ieee, doi=false, isbn=false]{biblatex}
\addbibresource{bib.bib}
\usepackage[nottoc]{tocbibind}
\usepgfplotslibrary{external}


%%%%%%%%%%%%% Images %%%%%%%%%%%%%%%%%%%%%%%%%%%%%%%%%%%%%%%%%%%%%%%%%%%%%%%%%%%

\graphicspath{{img/}}

\newcommand{\includetexpdf}[3]%
{ %
	{#2} % fontsize
	\def\svgwidth{#3\columnwidth} % size
	\input{img/#1.pdf_tex} % path
}


%%%%%%%%% Fonts %%%%%%%%%%%%%%%%%%%%%%%%%%%%%%%%%%%%%%%%%%%%%%%%%%%%%%%%%%%%%%%%

\renewcommand{\familydefault}{\sfdefault}
%%\usepackage{mathptmx}
%%\usepackage[lining, scale = 0.9]{firasans}
%\usepackage[scale=0.87]{tgheros}
%\usepackage[]{sfmath}
%\usepackage[scale=0.87, lining]{firamono}

% ----- tx fonts -> italic boldmath -----
\usepackage[scale=0.94]{tgheros} % fit uppercase
\usepackage[tx]{sfmath}
%\usepackage[scale=0.94, lining]{firamono} % fit undercase


%% xcharter ---------------------------------------------------------------------
%\usepackage[sups, scaled=.96, osf]{XCharter} % 'osf' = oldstyle figures
%\linespread{1.04}
%\usepackage[scaled=0.937, osf]{FiraSans}
%\usepackage[scaled=0.88, osf]{FiraMono}
%%\usepackage[scale=0.89]{tgheros}
%%\usepackage[varqu,varl, scaled=1.04]{inconsolata} % sans serif typewr iter
%%\usepackage[xcharter,bigdelims,vvarbb,scaled=1.055]{newtxmath}  % bb from STIX
%\AtBeginDocument {
%	\DeclareSymbolFont{bch}{T1}{\rmdefault}{m}{n}
%	\DeclareMathDelimiter{(}{\mathopen}{bch}{'050}{largesymbols}{"00}
%	\DeclareMathDelimiter{)}{\mathopen}{bch}{'051}{largesymbols}{"01}
%}
%\addtokomafont{disposition}{\normalfont\sffamily\firamedium}
%
%\usepackage[charter,expert, cal=cmcal]{mathdesign}
%%\usepackage[mathcal]{eucal}
%
%
%%\usepackage[mathcal]{eucal}



%%%%%%%%% Mathematics %%%%%%%%%%%%%%%%%%%%%%%%%%%%%%%%%%%%%%%%%%%%%%%%%%%%%%%%%%

% Red Boxed Equation -----------------------------------------------------------
\newcommand\widecolourbox[1]{{\setlength\fboxrule{0.5pt}\setlength\fboxsep{3pt}\fcolorbox{red}{white}{\enspace#1\enspace }}}


% Theorems % enable "theorem" package to use!! ---------------------------------
\newtheoremstyle{thmStyle}
{\topsep} % Space above
{\topsep} % Space below
{\itshape} % Body font
{} % Indent amount
{\sffamily} % Theorem head font
%{\sffamily\firamedium} % Theorem head font
{:} % Punctuation after theorem head
{.5em} % Space after theorem head
{} % Theorem head spec (can be left empty, meaning `normal')
\theoremstyle{thmStyle} \newtheorem{example}{Example}
\theoremstyle{thmStyle} \newtheorem{remark}{Remark}
\theoremstyle{thmStyle} \newtheorem{definition}{Definition}
\theoremstyle{thmStyle} \newtheorem{lemma}{Lemma}
\theoremstyle{thmStyle} \newtheorem{corollary}{Corollary}
\theoremstyle{thmStyle} \newtheorem{theorem}{Theorem}

\setlength{\mathindent}{1em}

% Mathematische Opteratoren ----------------------------------------------------
%\DeclareMathOperator{\rg}{rg}
%\DeclareMathOperator{\diag}{diag}
%\DeclareMathOperator{\spur}{tr}
%\DeclareMathOperator{\mspan}{span}
%\DeclareMathOperator{\mim}{im}
%\DeclareMathOperator{\grad}{grad}

%% Vectors
%\renewcommand{\vec}[1]{\underline{#1}}
%\renewcommand{\vec}[1]{\vv{#1}}
%\renewcommand{\vec}[1]{\overrightarrow{#1}}
\renewcommand{\vec}{\vectorbold*} % using physics package

% Differential
\newcommand{\dif}{\, \text{d}}
\newcommand{\diff}{\text{d}}

% Double underline for results
\def\doubleunderline#1{\underline{\underline{#1}}}

% Colorbox empheq --------------------------------------------------------------
\newtcbox{\mathbox}[1][]{nobeforeafter, math upper, tcbox raise base, 
	enhanced, sharp corners, colback=white, colframe=red, left=0.5em, top=0.25em, right=0.5em, bottom=0.25em }
% drop fuzzy shadow,

% Math Display Spacing----------------------------------------------------------
\makeatletter
\g@addto@macro \normalsize {%
	\setlength\abovedisplayskip{0.3em plus 0.25em minus 0.1em}%
	\setlength\belowdisplayskip{0.3em plus 0.25em minus 0.1em}%
}
\makeatother



%%%%%%%%% Layout %%%%%%%%%%%%%%%%%%%%%%%%%%%%%%%%%%%%%%%%%%%%%%%%%%%%%%%%%%%%%%%

\renewcommand\floatpagefraction{.9}
\renewcommand\dblfloatpagefraction{.9} % for two column documents
\renewcommand\topfraction{.9}
\renewcommand\dbltopfraction{.9} % for two column documents
\renewcommand\bottomfraction{.9}
\renewcommand\textfraction{.1}   
\setcounter{totalnumber}{50}
\setcounter{topnumber}{50}
\setcounter{bottomnumber}{50}

%\renewcommand{\familydefault}{\sfdefault}
\setlength\multicolsep{0pt}

\addtolength{\parskip}{-0.4em}
\addtolength{\floatsep}{-0.4em}
\addtolength{\textfloatsep}{-0.4em}
\addtolength{\intextsep}{-0.8em}

% Farben
\definecolor{hellblau}{rgb}{0.75,0.88,1}

\setlength{\columnseprule}{0.25pt}

\setlength{\footskip}{15pt}

% Save Space -------------------------------------------------------------------
\tolerance=2000 %Toleranz für Wortzwischenräume
\setlength{\emergencystretch}{20pt} %Zusätzliche Zeilendehnbarkeit in Notfällen

\linespread{0.95}

% ----- Sections Type 1 --------------------------------------------------------

%\makeatletter
%\renewcommand{\section}{\@startsection{section}{1}{0mm}%
%	{-1.2\baselineskip}{0.8\baselineskip}%
%	{\hrule depth 0.2pt width\columnwidth\hrule depth0.1em
%		width0.35\columnwidth\vspace*{0.6em}\Large\bfseries\sffamily}}
%\makeatother
%
%\makeatletter
%\renewcommand{\subsection}{\@startsection{subsection}{1}{0mm}%
%	{-1\baselineskip}{0.6\baselineskip}%
%	{\hrule depth 0.2pt width\columnwidth\hrule depth0.06em
%		width0.25\columnwidth\vspace*{0.6em}\large\bfseries\sffamily}}
%\makeatother
%
%\makeatletter
%\renewcommand{\subsubsection}{\@startsection{subsubsection}{1}{0mm}%
%	{-1\baselineskip}{0.4\baselineskip}%
%	{\hrule depth 0.2pt width\columnwidth\vspace*{0.6em}\normalsize\bfseries\sffamily}}
%\makeatother

% Sections Type 2 --------------------------------------------------------------

% Chapter
\colorlet{chapterbackground}{orange!80}
\addtokomafont{chapter}{\color{black}}
\makeatletter
\renewcommand\chapterlinesformat[3]{%
	\colorbox{chapterbackground}{%
		\parbox[c][\dimexpr\totalheight+0.1em]{\dimexpr\linewidth-2\fboxsep}{%
			\raggedchapter%
			\@hangfrom{#2}#3%
	}}%
}
\renewcommand\chapterlineswithprefixformat[3]{%
	\colorbox{chapterbackground}{%
		\parbox[c][\dimexpr\totalheight+1em]{\dimexpr\linewidth-2\fboxsep}{%
			\raggedchapter%
			#2#3%
	}}%
}
\makeatother

\usepackage{etoolbox}
\makeatletter
\patchcmd{\scr@startchapter}{\if@openright\cleardoublepage\else\clearpage\fi}{}{}{}
\makeatother


\colorlet{sectioncolor}{blue!50}
\colorlet{subsectioncolor}{blue!30}
\colorlet{subsubsectioncolor}{blue!15}
\makeatletter
\renewcommand\sectionlinesformat[4]{%
	\hspace{#2}%
	\colorbox{#1color}{%
		\parbox[c]{\dimexpr\columnwidth-2\fboxsep-#2\relax}{%
			\raggedsection\color{black}\@hangfrom{#3}{#4}%
		}}}
\makeatother
%\RedeclareSectionCommand[indent=0]{section}
%\renewcommand\raggedsection{\centering}
\RedeclareSectionCommand[beforeskip=1em, afterskip=0.001em]{chapter}
\RedeclareSectionCommand[beforeskip=0em, afterskip=0.001em]{section}
\RedeclareSectionCommand[beforeskip=0em, afterskip=0.001em]{subsection}
\RedeclareSectionCommand[beforeskip=0em, afterskip=0.001em]{subsubsection}

\setkomafont{chapter}{\Large}
\setkomafont{section}{\large}
\setkomafont{subsection}{\normalsize}
\setkomafont{subsubsection}{\normalsize}

% Itemize ----------------------------------------------------------------------
\setlist[itemize]{noitemsep, topsep=0em, leftmargin=2em, after=\vspace{0.2em}, before=\vspace{0.0em}}
\setlist[enumerate]{noitemsep, topsep=0em, leftmargin=2em, after=\vspace{0.2em}, before=\vspace{0.0em}}

% Paragraph --------------------------------------------------------------------
\newcommand{\parainline}[1]{\textsf{\textbf{#1:}}}

\makeatletter
\renewcommand{\paragraph}{%
	\@startsection{paragraph}{4}%
	{\z@}{0ex \@plus 0.5ex \@minus 0.2ex}{-0.5em}% %1ex: Abstand vertikal, plus/minus space streching, {...} = horizontal space
	{\normalfont\normalsize\bfseries\sffamily}%
%	{\normalfont\normalsize\bfseries\sffamily\firamedium}%
}
\makeatother

\reversemarginpar
\newcommand{\note}[1]{\protect\marginnote{\textit{{\normalsize\textbf{ #1}}}}}

%%%%%%%%%%%%% Custom Commands %%%%%%%%%%%%%%%%%%%%%%%%%%%%%%%%%%%%%%%%%%%%%%%%%%

% Commands ---------------------------------------------------------------------
\newcommand{\mdefinition}[1]{\textbf{\textsf{Def.}} #1}
\newcommand{\mbeispiel}[1]{\textit{\textsf{Bsp.}} #1}
\newcommand{\tipp}{\textbf{\textsf{Tipp: }}}
\newcommand{\emfo}[1]{\setlength{\fboxrule}{1pt}\fcolorbox{Red}{White}{#1}} % emphasize formulas

\newcommand{\sectionnote}[1]{\hfill \normalsize\normalfont{#1}}
\newcommand{\refbook}[1]{\normalfont\normalsize{\hfill p.\ #1}}

\newcommand{\matlab}{\textsc{Matlab}\textsuperscript{\tiny{\textregistered}}}


%%%%%%%%%%%% PDF %%%%%%%%%%%%%%%%%%%%%%%%%%%%%%%%%%%%%%%%%%%%%%%%%%%%%%%%%%%%%%%

\hypersetup{
	hidelinks=true,
	%	frenchlinks=true,
	%	linkcolor=black,
	%	filecolor=black,      
	%	urlcolor=black,
	%	citecolor=black,
	%	allcolors=black,
	%	allbordercolors=red,
	%pdfpagemode=FullScreen,
	pdftitle= {\mytitle},
	pdfauthor={\myauthor},
	%pdfkeywords={},
%	pdfcreator={Some fancy PDF-Creator...},
	bookmarksnumbered=true
}
\urlstyle{same}

%%%%%%%%%%%% Code %%%%%%%%%%%%%%%%%%%%%%%%%%%%%%%%%%%%%%%%%%%%%%%%%%%%%%%%%%%%%%

%\lstset{
%	backgroundcolor=\color{white},
%	tabsize=2,
%	rulecolor=,
%	language=C++,
%	%basicstyle=\scriptsize, % noch kleinerer Code
%	basicstyle=\ttfamily, %mittlerer Code
%	upquote=true,
%	aboveskip={0.3\baselineskip},
%	columns=fixed,
%	showstringspaces=false,
%	extendedchars=true,
%	breaklines=true,
%	prebreak = \raisebox{0ex}[0ex][0ex]{\ensuremath{\hookleftarrow}},
%	frame=single,
%	showtabs=false,
%	showspaces=false,
%	showstringspaces=false,
%	identifierstyle=\ttfamily,
%	keywordstyle=\color[rgb]{0,0,1},
%	commentstyle=\color[rgb]{0.133,0.545,0.133},
%	stringstyle=\color[rgb]{0.627,0.126,0.941},
%	numbers=left,
%	numberstyle=\tiny,
%	stepnumber=1,
%	numbersep=5pt,
%	belowskip=1em
%}



%%%%%%%%%%%%% Colors %%%%%%%%%%%%%%%%%%%%%%%%%%%%%%%%%%%%%%%%%%%%%%%%%%%%%%%%%%%


\newcommand{\myred}[1]{\textcolor{red}{#1}}
\newcommand{\myblue}[1]{\textcolor{blue}{#1}}
\newcommand{\mygreen}[1]{\textcolor{green}{#1}}
\newcommand{\myorange}[1]{\textcolor{orange}{#1}}
\newcommand{\mygray}[1]{\textcolor{gray}{#1}}

\newcommand{\speed}[1]{\myorange{#1}}

\newenvironment{notrelevant}{\color{gray}}{\ignorespacesafterend}



%%%%%%%%%%%%% Webpages %%%%%%%%%%%%%%%%%%%%%%%%%%%%%%%%%%%%%%%%%%%%%%%%%%%%%%%%%

% note: Squezing space in Latex http://www-h.eng.cam.ac.uk/help/tpl/textprocessing/squeeze.html

\begin{document}
\begin{multicols*}{3}

% Title
{\bfseries\sffamily\LARGE\mytitle} \vspace{0.35em}  \hfill \myauthor, \today \vspace{0.5em}

%\tableofcontents

%%%%%%%%%% INPUT %%%%%%%%%%%%%%%%%%%%%%%%%%%%%%%%%%%%%%%%%%%%%%%%%%%%%%%%%%%%%%%

\hfill

\chapter{Data Modeling with Relational Databases}
Learn about fundamentals of how to do relational data modeling by studying:
\begin{itemize}
\item Normalization
\item DeNormalization
\item Fact/Dimension tables
\item Different schema models
\end{itemize}

Some definitions:
\begin{itemize}
\item \textbf{Database} A set of related data and the way it is organized.
\item \textbf{Database Manegment System} Computer software that allows users to interact with databases. Provides access to all of the data.

\end{itemize}

\subsubsection*{RULE 1: The Information Rule}
All information in a relational database is represented explicitly at the logical level and in exactly one way - \textbf{by values in tables.}

\section{Importance of Relational Databses}
\begin{itemize}
\item \textbf{Standardization of data model}: Once your data is transformed into the rows and columns format, your data is standardized and you can query it with SQL

\item \textbf{Flexibility in adding and altering tables}: Relational databases gives you flexibility to add tables, alter tables, add and remove data.

\item \textbf{Data Integrity}: Data Integrity is the backbone of using a relational database. ACID

\item \textbf{Structured Query Language (SQL)}: A standard language can be used to access the data with a predefined language.

\item \textbf{Simplicity}: Data is systematically stored and modeled in tabular format.

\item \textbf{Intuitive Organization}: The spreadsheet format is intuitive but intuitive to data modeling in relational databases.
\end{itemize}

\section{OLAP Vs OLTP}
Databases \subsubsection{Online Analytical Processing OLAP}
Databases optimized for these workloads allow for \textbf{complex analytical and ad-hoc queries}. 
\textbf{These type of databases are optimized for reads.}

\subsubsection{Online Transactional Processing OLTP}
Databases optimized for these workloads allow for \textbf{less queries in large volume}. 
Types of queries: \textbf{read, insert, update, delete}

\subsubsection*{KEY DIFFERENCE}
The key to remember the difference between OLAP and OLTP is analytics (A) vs transactions (T). If you want to get the price of a shoe then you are using OLTP (this has very little or no aggregations). If you want to know the total stock of shoes a particular store sold, then this requires using OLAP (since this will require aggregations).


\section{Normalization}
To reduce data redundancy and increase data integrity

\paragraph{Definition}: The process of structuring a relational database in accordance with a series of \textbf{normal forms} in order to reduce data redundancy and increase data integrity. 

Normalization organizes the columns and tables in a database to ensure that their \textbf{dependencies are properly enforced by the database integrity constraints.}

\subsubsection*{IMPORTANT}
No need of extra copies of data. (Redundancy)
We want data to be in one place which is source of truth (Integrity)
More copies of data means more likely it is not properly updated.

\subsection{Objectives of Normalization}
\begin{itemize}
\item To Free database from unwanted insertions, updates and deletion dependencies. (If I wanna update my data, I would want to update it in ONE place).

\item To reduce need for re-factoring the database as new types of data are introduced.

\item To make relational model more informative to users. (Model real life concepts)

\item To make the database neutral to query statistics. (Not make table for specific queries, all queries should be gathered over time).

\end{itemize}

\subsection{Normal Forms}

\subsubsection{First Normal Form 1NF}
\begin{itemize}
\item Atomic values: Each cell contains unique single values
\item Be able to add data without altering tables.
\item Separate different relations into different tables. Customer and Sales separate table.
\item Keep relationships between tables together with foreign keys. (Ability to relate separate relations with each other)

\end{itemize}
\subsubsection{Second Normal Form 2NF}
\begin{itemize}
\item Have reached 1NF
\item All columns in a table must rely only on Primary Key. Should not need 2 columns to reach the third. IF this is the case, then might need to split tables.

If the column present in the table does not describe the entity, it may not fit the table.

\end{itemize}

\subsubsection{Third Normal Form 3NF}
\begin{itemize}
\item Must be in 2NF
\item No transitive dependencies => Remember, transitive dependencies you are trying to maintain is that to get from A-> C, you want to avoid going through B.
\end{itemize}

\subsubsection*{When to use 3NF}
When you want to update data, we want to be able to do in just 1 place. We want to avoid updating the table in the Customers Detail table

\section{Denormalization}
Must be done in read heavy workloads to increase performance.

\paragraph{Definition} The process of trying to \textbf{improve the read performance of a database at the expense of losing some write performance} by adding redundant copies of data. 

\begin{itemize}
\item Process of Denormalization is all about performance (read)
\item \textbf{JOINs allow for flexibility but JOINS are really slow.} Therefore, sometimes, heavy read tables might need some duplicate data to avoid unnecessary JOINS.
\item While dealing with Heavy Reads, you might want to think about Denormalization.
\item Denormalization always comes after Normalization is done.
\item Denormalization always needs more space.
\item SELECTs will be faster => INSERT, UPDATE and DELETE are slower after Denormalization.
\end{itemize}

\subsection{Fact and Dimension Tables}
A general rule of thumb to help divide information under different categories of tables:

\begin{itemize}
\item \textbf{Fact table} consists of the measurements, metrics or facts of a business process. They are normally int or numbers. [How many of the products were bought?]

\item \textbf{Dimension Table} A structure that categorizes facts and measures in order to enable users to answer business questions. Dimensions are people, products, place and time. [Where/When/what product was bought?]

\end{itemize}

\subsubsection*{IMPLEMENTING DIFFERENT SCHEMAS}
Based on facts and tables, we can design \textbf{data mart schemas} in two different styles:
\begin{itemize}
\item Star Schema
\item Snowflake Schema
\end{itemize}

\section{Star Schema}
\begin{itemize}
\item Simplest style of data mart schemas.
\item Star Schema consists of 1 or more fact tables.
\item Fact tables reference any number of dimension tables.
\item Why Star? Fact Table is in the center.
\item Dimension table surround a Fact table they want to reference.
\end{itemize}

\subsubsection*{BENEFITS of STAR SCHEMA}
\begin{itemize}
\item Can denormalize tables
\item Simplifies queries
\item Fast aggregations [The work of creating tables to make easily accessible aggregations is push out of application development]

\subsubsection*{DRAWBACKS of STAR SCHEMA}
\item Denormalization issues
\item Data integrity - Duplicated data
\item Decrease query flexibility (Now modeling to query)
\item Many to many relationships are hard to support.
\end{itemize}

\section{Snowflake Schema}
A snowflake schema has 
\begin{itemize}
\item Multiple levels of relationships
\item child tables having multiple parents
\item A complex shape emerges when dimensions of snowflake schema are elaborated.
\end{itemize}

\subsubsection*{STAR Vs SNOWFLAKE}
\begin{itemize}
\item Star schema is a very simplified and special case of \textbf{Snowflake} schema.
\item One to Many relationships are NOT allowed in Star Schema while Snowflake schema allows it.
\item Snowflake Schema is \textbf{MORE NORMALIZED} than Star schema but only in 1NF or 2NF.
\end{itemize}

\section{Data Definition And Constraints}
\begin{itemize}
\item NOT NULL: Column cannot contain a NULL value. (Can add NOT NULL to multiple columns in case of composite keys)
\item UNIQUE: Specifies that the data accross all the rows in one column are unique.(Also used for multiple columns)

\item PRIMARY KEY: defined on a single column, and every table should contain a primary key. The values in this column uniquely identify the rows in the table. If a group of columns are defined as a primary key, they are called a composite key. That means the combination of values in these columns will uniquely identify the rows in the table. By default, the PRIMARY KEY constraint has the unique and not null constraint built into it.
\end{itemize}


\section{UPSERT}
\begin{itemize}
\item In RDBMS language, the term upsert refers to the idea of inserting a new row in an existing table, or updating the row if it already exists in the table. The action of updating or inserting has been described as "upsert".

\item The way this is handled in PostgreSQL is by using the INSERT statement in combination with the ON CONFLICT clause.

\item The INSERT statement adds in new rows within the table. The values associated with specific target columns can be added in any order.

\item ON CONFLICT DO NOTHING: Want to try an update, but in case of a conflict do nothing

\item ON CONFLICT DO UPDATE: Want to try an update but if there is a conflict on column specified. Update.

\end{itemize}

%%%%%%%%%% REFERENCES %%%%%%%%%%%%%%%%%%%%%%%%%%%%%%%%%%%%%%%%%%%%%%%%%%%%%%%%%%

%\newpage
%\columnbreak
\nocite{*}
\printbibliography

\end{multicols*}
\end{document}
