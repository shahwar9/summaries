%%%%%%%% Packages %%%%%%%%%%%%%%%%%%%%%%%%%%%%%%%%%%%%%%%%%%%%%%%%%%%%%%%%%%%%%%

\usepackage[top=0.5cm, left=0.5cm, right=0.5cm, bottom=0.8cm, landscape]{geometry}
\usepackage[T1]{fontenc}
\usepackage[utf8]{inputenc} 
%\usepackage[ngerman]{babel}
\usepackage[english]{babel} 
%\usepackage[charter, expert, cal=cmcal]{mathdesign}
\usepackage{array}
%\usepackage{listings}
\usepackage{enumerate}
\usepackage{ifthen,calc}
\usepackage{empheq}
\usepackage{wrapfig}
\usepackage{amsmath, amssymb, amsthm}
\usepackage{multicol}
\usepackage[svgnames,table,hyperref, dvipsnames]{xcolor}
\usepackage{graphicx}
\usepackage{multirow}
\usepackage{hyperref}
\usepackage{verbatim}
\usepackage{enumitem}
\usepackage{textcomp}
\usepackage{parskip}
\usepackage{blindtext}
\usepackage{floatflt}
\usepackage{float}
\usepackage{booktabs}
\usepackage[fulladjust]{marginnote}
\usepackage{longtable}
\usepackage{xtab,afterpage}
\usepackage{xcolor}
\usepackage{tikz}
\usepackage{accents}
\usepackage{tabularx}
\usepackage{hyperref}
\usepackage{bookmark}
\usepackage{siunitx}
\usepackage[most]{tcolorbox}
\usepackage{titling}
\usepackage[a]{esvect} % vectors, options a--h
\usepackage{tikz}
\usepackage{pgfplots}
\usepackage{physics}
\usepackage{booktabs}
\usepackage{empheq}
\usepackage{csquotes}
\usepackage{microtype}
\usepackage[style=ieee, doi=false, isbn=false]{biblatex}
\addbibresource{bib.bib}
\usepackage[nottoc]{tocbibind}
\usepgfplotslibrary{external}


%%%%%%%%%%%%% Images %%%%%%%%%%%%%%%%%%%%%%%%%%%%%%%%%%%%%%%%%%%%%%%%%%%%%%%%%%%

\graphicspath{{img/}}

\newcommand{\includetexpdf}[3]%
{ %
	{#2} % fontsize
	\def\svgwidth{#3\columnwidth} % size
	\input{img/#1.pdf_tex} % path
}


%%%%%%%%% Fonts %%%%%%%%%%%%%%%%%%%%%%%%%%%%%%%%%%%%%%%%%%%%%%%%%%%%%%%%%%%%%%%%

\renewcommand{\familydefault}{\sfdefault}
%%\usepackage{mathptmx}
%%\usepackage[lining, scale = 0.9]{firasans}
%\usepackage[scale=0.87]{tgheros}
%\usepackage[]{sfmath}
%\usepackage[scale=0.87, lining]{firamono}

% ----- tx fonts -> italic boldmath -----
\usepackage[scale=0.94]{tgheros} % fit uppercase
\usepackage[tx]{sfmath}
%\usepackage[scale=0.94, lining]{firamono} % fit undercase


%% xcharter ---------------------------------------------------------------------
%\usepackage[sups, scaled=.96, osf]{XCharter} % 'osf' = oldstyle figures
%\linespread{1.04}
%\usepackage[scaled=0.937, osf]{FiraSans}
%\usepackage[scaled=0.88, osf]{FiraMono}
%%\usepackage[scale=0.89]{tgheros}
%%\usepackage[varqu,varl, scaled=1.04]{inconsolata} % sans serif typewr iter
%%\usepackage[xcharter,bigdelims,vvarbb,scaled=1.055]{newtxmath}  % bb from STIX
%\AtBeginDocument {
%	\DeclareSymbolFont{bch}{T1}{\rmdefault}{m}{n}
%	\DeclareMathDelimiter{(}{\mathopen}{bch}{'050}{largesymbols}{"00}
%	\DeclareMathDelimiter{)}{\mathopen}{bch}{'051}{largesymbols}{"01}
%}
%\addtokomafont{disposition}{\normalfont\sffamily\firamedium}
%
%\usepackage[charter,expert, cal=cmcal]{mathdesign}
%%\usepackage[mathcal]{eucal}
%
%
%%\usepackage[mathcal]{eucal}



%%%%%%%%% Mathematics %%%%%%%%%%%%%%%%%%%%%%%%%%%%%%%%%%%%%%%%%%%%%%%%%%%%%%%%%%

% Red Boxed Equation -----------------------------------------------------------
\newcommand\widecolourbox[1]{{\setlength\fboxrule{0.5pt}\setlength\fboxsep{3pt}\fcolorbox{red}{white}{\enspace#1\enspace }}}


% Theorems % enable "theorem" package to use!! ---------------------------------
\newtheoremstyle{thmStyle}
{\topsep} % Space above
{\topsep} % Space below
{\itshape} % Body font
{} % Indent amount
{\sffamily} % Theorem head font
%{\sffamily\firamedium} % Theorem head font
{:} % Punctuation after theorem head
{.5em} % Space after theorem head
{} % Theorem head spec (can be left empty, meaning `normal')
\theoremstyle{thmStyle} \newtheorem{example}{Example}
\theoremstyle{thmStyle} \newtheorem{remark}{Remark}
\theoremstyle{thmStyle} \newtheorem{definition}{Definition}
\theoremstyle{thmStyle} \newtheorem{lemma}{Lemma}
\theoremstyle{thmStyle} \newtheorem{corollary}{Corollary}
\theoremstyle{thmStyle} \newtheorem{theorem}{Theorem}

\setlength{\mathindent}{1em}

% Mathematische Opteratoren ----------------------------------------------------
%\DeclareMathOperator{\rg}{rg}
%\DeclareMathOperator{\diag}{diag}
%\DeclareMathOperator{\spur}{tr}
%\DeclareMathOperator{\mspan}{span}
%\DeclareMathOperator{\mim}{im}
%\DeclareMathOperator{\grad}{grad}

%% Vectors
%\renewcommand{\vec}[1]{\underline{#1}}
%\renewcommand{\vec}[1]{\vv{#1}}
%\renewcommand{\vec}[1]{\overrightarrow{#1}}
\renewcommand{\vec}{\vectorbold*} % using physics package

% Differential
\newcommand{\dif}{\, \text{d}}
\newcommand{\diff}{\text{d}}

% Double underline for results
\def\doubleunderline#1{\underline{\underline{#1}}}

% Colorbox empheq --------------------------------------------------------------
\newtcbox{\mathbox}[1][]{nobeforeafter, math upper, tcbox raise base, 
	enhanced, sharp corners, colback=white, colframe=red, left=0.5em, top=0.25em, right=0.5em, bottom=0.25em }
% drop fuzzy shadow,

% Math Display Spacing----------------------------------------------------------
\makeatletter
\g@addto@macro \normalsize {%
	\setlength\abovedisplayskip{0.3em plus 0.25em minus 0.1em}%
	\setlength\belowdisplayskip{0.3em plus 0.25em minus 0.1em}%
}
\makeatother



%%%%%%%%% Layout %%%%%%%%%%%%%%%%%%%%%%%%%%%%%%%%%%%%%%%%%%%%%%%%%%%%%%%%%%%%%%%

\renewcommand\floatpagefraction{.9}
\renewcommand\dblfloatpagefraction{.9} % for two column documents
\renewcommand\topfraction{.9}
\renewcommand\dbltopfraction{.9} % for two column documents
\renewcommand\bottomfraction{.9}
\renewcommand\textfraction{.1}   
\setcounter{totalnumber}{50}
\setcounter{topnumber}{50}
\setcounter{bottomnumber}{50}

%\renewcommand{\familydefault}{\sfdefault}
\setlength\multicolsep{0pt}

\addtolength{\parskip}{-0.4em}
\addtolength{\floatsep}{-0.4em}
\addtolength{\textfloatsep}{-0.4em}
\addtolength{\intextsep}{-0.8em}

% Farben
\definecolor{hellblau}{rgb}{0.75,0.88,1}

\setlength{\columnseprule}{0.25pt}

\setlength{\footskip}{15pt}

% Save Space -------------------------------------------------------------------
\tolerance=2000 %Toleranz für Wortzwischenräume
\setlength{\emergencystretch}{20pt} %Zusätzliche Zeilendehnbarkeit in Notfällen

\linespread{0.95}

% ----- Sections Type 1 --------------------------------------------------------

%\makeatletter
%\renewcommand{\section}{\@startsection{section}{1}{0mm}%
%	{-1.2\baselineskip}{0.8\baselineskip}%
%	{\hrule depth 0.2pt width\columnwidth\hrule depth0.1em
%		width0.35\columnwidth\vspace*{0.6em}\Large\bfseries\sffamily}}
%\makeatother
%
%\makeatletter
%\renewcommand{\subsection}{\@startsection{subsection}{1}{0mm}%
%	{-1\baselineskip}{0.6\baselineskip}%
%	{\hrule depth 0.2pt width\columnwidth\hrule depth0.06em
%		width0.25\columnwidth\vspace*{0.6em}\large\bfseries\sffamily}}
%\makeatother
%
%\makeatletter
%\renewcommand{\subsubsection}{\@startsection{subsubsection}{1}{0mm}%
%	{-1\baselineskip}{0.4\baselineskip}%
%	{\hrule depth 0.2pt width\columnwidth\vspace*{0.6em}\normalsize\bfseries\sffamily}}
%\makeatother

% Sections Type 2 --------------------------------------------------------------

% Chapter
\colorlet{chapterbackground}{orange!80}
\addtokomafont{chapter}{\color{black}}
\makeatletter
\renewcommand\chapterlinesformat[3]{%
	\colorbox{chapterbackground}{%
		\parbox[c][\dimexpr\totalheight+0.1em]{\dimexpr\linewidth-2\fboxsep}{%
			\raggedchapter%
			\@hangfrom{#2}#3%
	}}%
}
\renewcommand\chapterlineswithprefixformat[3]{%
	\colorbox{chapterbackground}{%
		\parbox[c][\dimexpr\totalheight+1em]{\dimexpr\linewidth-2\fboxsep}{%
			\raggedchapter%
			#2#3%
	}}%
}
\makeatother

\usepackage{etoolbox}
\makeatletter
\patchcmd{\scr@startchapter}{\if@openright\cleardoublepage\else\clearpage\fi}{}{}{}
\makeatother


\colorlet{sectioncolor}{blue!50}
\colorlet{subsectioncolor}{blue!30}
\colorlet{subsubsectioncolor}{blue!15}
\makeatletter
\renewcommand\sectionlinesformat[4]{%
	\hspace{#2}%
	\colorbox{#1color}{%
		\parbox[c]{\dimexpr\columnwidth-2\fboxsep-#2\relax}{%
			\raggedsection\color{black}\@hangfrom{#3}{#4}%
		}}}
\makeatother
%\RedeclareSectionCommand[indent=0]{section}
%\renewcommand\raggedsection{\centering}
\RedeclareSectionCommand[beforeskip=1em, afterskip=0.001em]{chapter}
\RedeclareSectionCommand[beforeskip=0em, afterskip=0.001em]{section}
\RedeclareSectionCommand[beforeskip=0em, afterskip=0.001em]{subsection}
\RedeclareSectionCommand[beforeskip=0em, afterskip=0.001em]{subsubsection}

\setkomafont{chapter}{\Large}
\setkomafont{section}{\large}
\setkomafont{subsection}{\normalsize}
\setkomafont{subsubsection}{\normalsize}

% Itemize ----------------------------------------------------------------------
\setlist[itemize]{noitemsep, topsep=0em, leftmargin=2em, after=\vspace{0.2em}, before=\vspace{0.0em}}
\setlist[enumerate]{noitemsep, topsep=0em, leftmargin=2em, after=\vspace{0.2em}, before=\vspace{0.0em}}

% Paragraph --------------------------------------------------------------------
\newcommand{\parainline}[1]{\textsf{\textbf{#1:}}}

\makeatletter
\renewcommand{\paragraph}{%
	\@startsection{paragraph}{4}%
	{\z@}{0ex \@plus 0.5ex \@minus 0.2ex}{-0.5em}% %1ex: Abstand vertikal, plus/minus space streching, {...} = horizontal space
	{\normalfont\normalsize\bfseries\sffamily}%
%	{\normalfont\normalsize\bfseries\sffamily\firamedium}%
}
\makeatother

\reversemarginpar
\newcommand{\note}[1]{\protect\marginnote{\textit{{\normalsize\textbf{ #1}}}}}

%%%%%%%%%%%%% Custom Commands %%%%%%%%%%%%%%%%%%%%%%%%%%%%%%%%%%%%%%%%%%%%%%%%%%

% Commands ---------------------------------------------------------------------
\newcommand{\mdefinition}[1]{\textbf{\textsf{Def.}} #1}
\newcommand{\mbeispiel}[1]{\textit{\textsf{Bsp.}} #1}
\newcommand{\tipp}{\textbf{\textsf{Tipp: }}}
\newcommand{\emfo}[1]{\setlength{\fboxrule}{1pt}\fcolorbox{Red}{White}{#1}} % emphasize formulas

\newcommand{\sectionnote}[1]{\hfill \normalsize\normalfont{#1}}
\newcommand{\refbook}[1]{\normalfont\normalsize{\hfill p.\ #1}}

\newcommand{\matlab}{\textsc{Matlab}\textsuperscript{\tiny{\textregistered}}}


%%%%%%%%%%%% PDF %%%%%%%%%%%%%%%%%%%%%%%%%%%%%%%%%%%%%%%%%%%%%%%%%%%%%%%%%%%%%%%

\hypersetup{
	hidelinks=true,
	%	frenchlinks=true,
	%	linkcolor=black,
	%	filecolor=black,      
	%	urlcolor=black,
	%	citecolor=black,
	%	allcolors=black,
	%	allbordercolors=red,
	%pdfpagemode=FullScreen,
	pdftitle= {\mytitle},
	pdfauthor={\myauthor},
	%pdfkeywords={},
%	pdfcreator={Some fancy PDF-Creator...},
	bookmarksnumbered=true
}
\urlstyle{same}

%%%%%%%%%%%% Code %%%%%%%%%%%%%%%%%%%%%%%%%%%%%%%%%%%%%%%%%%%%%%%%%%%%%%%%%%%%%%

%\lstset{
%	backgroundcolor=\color{white},
%	tabsize=2,
%	rulecolor=,
%	language=C++,
%	%basicstyle=\scriptsize, % noch kleinerer Code
%	basicstyle=\ttfamily, %mittlerer Code
%	upquote=true,
%	aboveskip={0.3\baselineskip},
%	columns=fixed,
%	showstringspaces=false,
%	extendedchars=true,
%	breaklines=true,
%	prebreak = \raisebox{0ex}[0ex][0ex]{\ensuremath{\hookleftarrow}},
%	frame=single,
%	showtabs=false,
%	showspaces=false,
%	showstringspaces=false,
%	identifierstyle=\ttfamily,
%	keywordstyle=\color[rgb]{0,0,1},
%	commentstyle=\color[rgb]{0.133,0.545,0.133},
%	stringstyle=\color[rgb]{0.627,0.126,0.941},
%	numbers=left,
%	numberstyle=\tiny,
%	stepnumber=1,
%	numbersep=5pt,
%	belowskip=1em
%}



%%%%%%%%%%%%% Colors %%%%%%%%%%%%%%%%%%%%%%%%%%%%%%%%%%%%%%%%%%%%%%%%%%%%%%%%%%%


\newcommand{\myred}[1]{\textcolor{red}{#1}}
\newcommand{\myblue}[1]{\textcolor{blue}{#1}}
\newcommand{\mygreen}[1]{\textcolor{green}{#1}}
\newcommand{\myorange}[1]{\textcolor{orange}{#1}}
\newcommand{\mygray}[1]{\textcolor{gray}{#1}}

\newcommand{\speed}[1]{\myorange{#1}}

\newenvironment{notrelevant}{\color{gray}}{\ignorespacesafterend}



%%%%%%%%%%%%% Webpages %%%%%%%%%%%%%%%%%%%%%%%%%%%%%%%%%%%%%%%%%%%%%%%%%%%%%%%%%

% note: Squezing space in Latex http://www-h.eng.cam.ac.uk/help/tpl/textprocessing/squeeze.html